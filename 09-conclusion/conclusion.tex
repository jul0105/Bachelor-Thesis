\documentclass[../report.tex]{subfiles}
\begin{document}
\chapter{Conclusion}
This chapter concludes this report with a summary of the work done, and what can still be done.
\section{Password security}
PAKE protocol provides the highest level of security for password-based authentication, and KHAPE provide the largest number of security guarantees among PAKE protocol. This makes KHAPE the most secure way to perform authentication with passwords.
However, passwords are still insecure by their very nature of being low-entropy secrets. Password-based authentication should ideally be replaced by more secure protocol such as WebAuthn.

\section{Final result}
Looking back at the goals defined in the specification, almost all of them have been achieved.
The comparison between the main PAKEs was extensive.
KHAPE has been chosen to be implemented, providing the first ever implementation of this protocol. %The implementation of the KHAPE library was successful, as well as the implementation of the use case.
A working prototype of an online password manager has been implemented for the use case.
And a performance test has been added to evaluate the efficiency of the developed library. In this regard, the developed library achieve almost similar performances than the OPAQUE's library with a slightly higher security level which is a great success.
The only thing that has not been done --- due to lack of time --- is the PAKE history. This section was supposed to provide an overview of the state of PAKE protocols --- from the most commons to the less known --- in a chronological order. In consequence, this report rarely mention any other PAKE protocols that is not one of the main protocols (EKE, SRP, OPAQUE and KHAPE). However, the explanations and comparison of the main PAKE protocols are much more significant than initially planned and provide --- in part --- the overview that the history should have provided, at least for the main protocols.

% - Résultat final (ce qui a été fait, ce qui n'a pas été fait, cahier du charge)

% Fait :
% - Description of EKE, SRP, OPAQUE and KHAPE
% - *Comprehensive* comparison table between these four PAKEs
% - First ever implement of the KHAPE protocol in any language
% - Implementation of an online password manager using the implemented KHAPE library
% - Performance evaluation of the KHAPE library and comparison with OPAQUE library
% Pas fait:
% - Historique

\section{Future work}
The KHAPE library has been implemented and is functional but multiple interesting concept has not been considered or implemented by lack of time.
\begin{itemize}
 \item Protection against timing attacks,
 \item Discharge passwords, and other sensitive value from memory directly after use,
 \item Password reset feature,
 \item Support for WebAssembly build (to be used directly in web browsers),
 \item Refactor the library to be generic and give the choice of the cryptographic primitives to the library's users,
 \item Performance comparison between 3DH and HMQV,
 \item Adapt the online password manager client for web browsers
\end{itemize}

% - Future work (ideas non implemented)
%   - continue to work and improve the KHAPE library
%   - use case is functional be can be improved to be really good (network, browser)
% - Comparison (result) with HMQV
% - Timing attacks consideration (protection)
% - Consider discharging password (and other secret values) from memory immediatly after use

% PAKE increase the security of passwords but they are still insecure. future: use other auth method
% \section{Personal conclusion}
\section{Personal conclusion}
I had much pleasure to do this work and to implement KHAPE and I will continue to improve it in order to propose a public library. Reading a large number of cryptographic paper was new to me but it was very interesting to see how the cryptographic literature evolved with time.
In terms of difficulties, KHAPE's non-committing encryption was hard to understand and implement for me. It took me quite some time to understand the concept and how it works and even more time to find a way to implement it in the library.
Other than that, I didn't face any big difficulties.
% Ideal cipher
% Not much

\end{document}