\documentclass[../report.tex]{subfiles}

\begin{document}

\chapter{Conclusion}
\section{Final result}
Almost all the goal defined in the specification has been achieved.
The comparison between the main PAKEs was extensive. The implementation of the KHAPE library was successful, as well as the implementation of the use case.
A performance test has been added to evaluate the developed library.

The only thing that has not be done, due to lack of time, is the PAKE history.


% - Résultat final (ce qui a été fait, ce qui n'a pas été fait, cahier du charge)


% Fait :
% - Description of EKE, SRP, OPAQUE and KHAPE
% - *Comprehensive* comparison table between these four PAKEs
% - First ever implement of the KHAPE protocol in any language
% - Implementation of an online password manager using the implemented KHAPE library
% - Performance evaluation of the KHAPE library and comparison with OPAQUE library

% Pas fait:
% - Historique


% \section{Difficulties}
% Ideal cipher
% Not much

\section{Future work}
The KHAPE library has been implemented and is functional but multiple thing could still be done.
\begin{itemize}
 \item Protection against timing attacks,
 \item Discharge password, and other sensitive value from memory directly after use,
 \item Performance comparison of 3DH with HMQV,
 \item Adapt the online password manager client for web browsers
\end{itemize}

% - Future work (ideas non implemented)
%   - continue to work and improve the KHAPE library
%   - use case is functional be can be improved to be really good (network, browser)

% - Comparison (result) with HMQV
% - Timing attacks consideration (protection)
% - Consider discharging password (and other secret values) from memory immediatly after use


% PAKE increase the security of passwords but they are still insecure. future: use other auth method

% \section{Personal conclusion}



\end{document}

