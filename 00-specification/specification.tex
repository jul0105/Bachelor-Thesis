\documentclass[../report.tex]{subfiles}

\begin{document}

\chapter{Specification}
\section*{Context}

% - very powerful cryptographic primitive
% - allow a server to auth a client without having to know or to store the client's password.
% - allow to share key ?
% - better than classical auth method

% - not used enough now a day in the industry
% - many old PAKE were patented or got broken. Which might have hurt the adoption of this primitive in the industry.




Password authenticated key exchange (PAKE) are very powerful cryptographic primitive. They allow a server to share a key with a client or to authenticate a client without having to know or to store the client's password.

For this reason, they provide better security guaranties for initializing a secure connection using password than usual mechanism where the password is transmitted to the server and then compared to an hash.

Despite its *superiority*, PAKEs are not implemented enough in the industry. Many old PAKE were patented or got broken which might have hurt the adoption of this primitive.

% In the recents years, modern PAKE have been proposed

\section*{Goal and objectives}

The goal of this project is to study the main existing PAKE and to understand their differences.

The goal of this project is to study the existing PAKE, including the old one

% - study existing PAKE
% - old one and moderne one: SRP, OPAQUE, KHAPE, EKE, OKE (cassé), variantes d'EKE (PAK, PPK, PAK-X,), SNAPI, PEKEP. (pas besoin d'aller trop dans les détails pour les ancêtres)
% - look if more PAKE exist

% - Study in details the main PAKE (EKE, SRP, OPAQUE, KHAPE)
% - understand their differences


% - Then choose a modern PAKE (OPAQUE or KHAPE)
% - and design a interesting use case were using a KAPE make more sense than using a classical auth method
% - Explain the advantages that the PAKE has over using a classical auth method

% - Then implement the choosen PAKE and the use case using the desired programing language


\section*{Deliverables}

% - implementation of a PAKE with the use case
% - report containing the state of the art of PAKE, use case explaination and implementation details

\end{document}

