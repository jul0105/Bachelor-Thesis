\documentclass[../report.tex]{subfiles}

\begin{document}

\chapter{Specification}
\section*{Context}

% - very powerful cryptographic primitive
% - allow a server to auth a client without having to know or to store the client's password.
% - allow to share key ?
% - better than classical auth method

% - not used enough now a day in the industry
% - many old PAKE were patented or got broken. Which might have hurt the adoption of this primitive in the industry.




Password authenticated key exchange (PAKE) are very powerful cryptographic primitive. They allow a server to share a key with a client or to authenticate a client without having to know or to store the client's password.
For this reason, they provide better security guaranties for initializing a secure connection using password than usual mechanism where the password is transmitted to the server and then compared to an hash.
Despite its theoretical superiority, PAKEs are not implemented enough in the industry. Many old PAKE were patented or got broken which might have hurt the adoption of this primitive.

% PAKEs stuggle de s'imposer dans l'industie. 

% In the recents years, modern PAKE have been proposed and effort are made to make PAKE an industry standard (?)

\section*{Goals}


% The goal of this project is to *have* an *overview* of the existing PAKE, including the old one (SRP, OPAQUE, KHAPE, EKE, OKE, EKE varient (PAK, PPK, PAK-X,), SNAPI, PEKEP and to look for less known PAKE)
% 
% Then study in details the *main* PAKE (EKE, SRP, OPAQUE, KHAPE) and understand their differences.
% 
% Once the analysis phase is done, one of the modern PAKE is chosen to be implemented. The choice is based on *qualities* of the PAKE and the existing implementation and/or the existence of standards or literature for the PAKE.
% 
% Then a interesting use case is designed were using a KAPE make more sense than using a classical auth method. The advantage of the PAKE over the classical auth method is detailed.
% 
% Then implement the choosen PAKE and the use case using the desired programing language


\begin{enumerate}
 \item Outline existing PAKE, including the old one. This include SRP, OPAQUE, KHAPE, EKE, OKE, EKE varient (PAK, PPK, PAK-X,), SNAPI and PEKEP. Also look for other less known PAKE.

 \item Study in details the main PAKE --- EKE, SRP, OPAQUE, KHAPE --- and understand their differences.

 \item Choose one of the modern PAKE to implement. The choice is based on the properties of the PAKE, the existence of implementations for this PAKE and/or the existence of standards for this PAKE.

 \item Design an interesting use case were using a KAPE is more appropriate than using a classical authentication method. The advantages of the PAKE are detailed in the report.

 \item Implement the chosen PAKE and the use case using the desired programming language
\end{enumerate}



% - study existing PAKE
% - old one and moderne one: SRP, OPAQUE, KHAPE, EKE, OKE (cassé), variantes d'EKE (PAK, PPK, PAK-X,), SNAPI, PEKEP. (pas besoin d'aller trop dans les détails pour les ancêtres)
% - look if more PAKE exist

% - Study in details the main PAKE (EKE, SRP, OPAQUE, KHAPE)
% - understand their differences


% - Then choose a modern PAKE (OPAQUE or KHAPE)
% - and design a interesting use case were using a KAPE make more sense than using a classical auth method
% - Explain the advantages that the PAKE has over using a classical auth method

% - Then implement the choosen PAKE and the use case using the desired programing language


\section*{Deliverables}

% - implementation of a PAKE with the use case
% - report containing the state of the art of PAKE, use case explaination and implementation details

\begin{itemize}
 \item Implementation of the chosen PAKE with the use case
 \item Report containing :
 \begin{itemize}
    \item State of the art of PAKEs,
    \item Description of the use case,
    \item Advantages of using a PAKE over a classical authentication method for this use case,
    \item Implementation details
 \end{itemize}
\end{itemize}
\end{document}

