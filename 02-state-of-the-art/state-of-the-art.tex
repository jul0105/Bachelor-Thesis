\documentclass[../report.tex]{subfiles}

\begin{document}

% Chapter State of the art
% - Concepts (before diving into the protocoles, it is important to understand some concepts)
% - Outlines of the KAPE landscape (all PAKEs, history, overview)
% - Mains PAKEs (differences, improvement, tableau comparatif, pas besoin de décrire les constructions)
%   - EKE
%   - SRP
%   - OPAQUE
%   - KHAPE
% - PAKE choice (and reasons)

% (Attack on PAKEs ?)


% Chapter OPAQUE or Chapter KHAPE (depending on the PAKE choice)
% - detail construction (protocol) of the choosen PAKE


\chapter{State of the art}
\section{Main PAKEs}

\subsection{OPAQUE}

% -------------------- 01 OPAQUE Paper ------------------

%%% Abstract :

% Construction :
% - "two modular constructions using an Oblivious PRF as a main tool"
%     - "builds a Strong aPAKE from any aPAKE (which in turn can be constructed from any PAKE [26])"
%     - "builds a Strong aPAKE from any authenticated key-exchange (AKE) protocol secure against KCI attacks."

% Performances :
% 3 messages
% 3 or 4 exponentiations

% Guarantees :
% - forward secrecy
% - "explicit mutual authentication"
% - PKI-free
% - "supports user-side password hardening"
% - "has a built-in facility for password-based storage-and-retrieval of secrets and credentials"
% - "accommodates a user-transparent server-side threshold implementation."



% -------------------- 06 Lets talk about PAKE ----------------

% Advantages of OPAQUE:
% - "can be implemented in any setting where Diffie-Hellman and discrete log (type) problems are hard"
% - "can be easily instantiated using efficient elliptic curves"
% - "does not reveal the salt". No pre-commutation attack
% - "works with any password hashing function"
% - "take load off the server", "use much strong security settings" like Argon2, scrypt, ... (with resource heavy parameters)
% - approx. the same amount of messages and exponentiations than SRP. "But since it can be implemented in more efficient settings, it’s likely to be a lot more efficient."
% - "security proof in a very strong model"

% - "internet draft proposal" (standard)
% - production-grade implementation ?



%%% Oblivious PRF design :
% 1. "The main problem with earlier PAKEs is the need to transmit the salt from a server to a client"
% 2. attackers can retrieve the salt to build an offline dictionary. This is what we call a pre-computation attack.
% 3. Password hash is done client-side but doesn't use the salt that the server store.
% 4. Oblivious PRF is used to calculate another salt called 'salt2'
% 5. Then the client calculate the secret key by hashing the password with the salt2
% - if the client enter an incorrect password, the output (secret key) will be very different and he won't be able to use the key for the next step 

% Guarantees :
% - client knows the password
% - server knows the salt
% - server doesn't learn password
% - client doesn't learn the salt stored on the server
% - only the client learn the secret key (the output)


%%% Key exchange design :
% - needs an authenticated key-exchange (HMQV)
% - needs two public/private keys pair. One for the server, one for the client.

% Register :
% - When a client want to register, the client generate a public/private key pair and encrypt the private key with the secret key (output of OPRF). The ciphertext (and the public key) is sent and stored in the server.
% "C = Encrypt(K, client's private_key | server's pulic_key)"

% Login :
% - When a client want to login, the server sends the stored ciphertext to the client
% - If the password entered by the client is correct, he get the correct secret key from the OPRF
% - He can then use the secret key to decrypt the ciphertext and retrieve his private key and server's public key
% - He then use these keys to run a authenticated key exchange with the server (like HMQV ?)


% ============================================================

\paragraph{design}




\end{document}
