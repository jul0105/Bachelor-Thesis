\documentclass[../report.tex]{subfiles}

\begin{document}

\chapter{\writingNotes{Use case}}
\label{cha:use_case}


\section{Online password manager}
Online password manager are among the most sensitive site out there because the compromise of user's data cascade into numerous account compromisation on other services such as email account, social media, banking, etc.

Using an asymmetric PAKE for an online password manager make a lot of sense because the client doesn't have to disclose it's master password to the password manager host. In other words, the client doesn't have to trust the password manager host to not decrypt its personal data and/or leak the master password (or any other intentional or unintentional miss-handling).

In fact multiple well known online password manager such as iCloud Key Vault or 1Password use aPAKE (SRP).

\section{Other use case}

More generally, using an aPAKE makes a lot of sense on application where the server-side stored user's data shouldn't be visible to the server (server doesn't process the data, online backup, online wallet?, secure vault, password manager, etc.). This is archived with encryption and so require an encryption key for the client.
Depending on the client, it is not feasable to store an additional symmetric key because it has to be securely stored (see HSM) which cause problem of portability and key recovery. For example, for an online encrypted backup of a laptop or smartphone, if the user loose its device, he cannot retrieve his online backup because the encryption key is stored on its lost device.

For portability, the encryption key is typically derived from the user's password --- the same password that he uses to authenticate with the server (you could require that the user input two differents passwords but this is generally avoided because of bad user experience). Using a classical authentication method, the server store the user's encrypted sensible data AND also process the password in cleartext which is used to compute the encryption key. This void all the security of encrypting the sensitive data in the first place because the server --- or an malicious party who compromised the server --- could store the cleartext password, compute the encryption key and decrypt the sensitive user's data.

This is the reason why aPAKEs are very interesting in these case senario. The server NEVER see the user's password so he cannot use it to decrypt user's data.

\section{Design}

client ---uid, OPRF---> server

client <---e, Y, OPRF--- server

client ---X, t1 ---> server

client <---t2, data--- server


% How to decrypt data (password manager's data) ?
% - password in KDF
% - using initial OPRF output
% - compute a new OPRF (different salt, same password ?)

\end{document}

