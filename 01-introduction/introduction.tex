\documentclass[../report.tex]{subfiles}

\begin{document}

% 1. Summary of the problem
%     1. Classical auth methods(, vulnerability and attacks)
%     2. PAKE
% 2. Generalities
%     - Contexte du travail, volume d'heure, dates, temps plein, etc...


\chapter{Introduction}


% ------ 06 A Few Thoughts on Cryptographic Engineering -------------
% Classical auth methods : OK

% traditional way to auth
% hash user's password and store them in a database
% connection between client and server should be secured (HTTPS, TLS)
% precomputed hash attack (raimbow tables)
% many way to do it and lot's of improvment 
    % using memory-hard password hashing function (scrypt, Argon2, etc..) (KDF ?)
    % using salt
    % using pepper
% all these approach have the same weakness:

% When the user wants to login, he has to send its cleartext password to the server in order for the server to authenticate the user.
% "This requirement can lead to disaster if your server is ever persistently compromised or if your developers make a simple mistake"



% PAKE at the rescue :

% (PAKE "provide protection for the client's password")
% "A stronger version of this guarantee can be stated as follows: after a login attempt (valid, or invalid) both the client and server should learn only whether the client’s password matched the server’s expected value, and no additional information. This is a powerful guarantee. In fact, it’s not dissimilar to what we ask for from a zero knowledge proof"
% PAKE allow to .........

% "many people dont want to run a key exchange protocol. They just want to verify that a user knows a password."
% "The great thing about PAKE is that the simpler “login only” use-case is easy to achieve. If I have a standard PAKE protocol that allows a client and server to agree on a shared key K if (and only if) the client knows the right password, then all we need add is a simple check that both parties have arrived at the same key."



% Why PAKE have almost no adoption :

% PAKE provides "an obvious security benefit compared to the naive approach we use to log into servers today." 
% PAKE are old. "they have been known since 1992"

% "Almost no adoption"
% "it's much easier to put a password form onto a web page than it is to do dancy crypto in the browser"
% But "even native app rarely implement PAKE for their logins"

% patents ("though most of these are expired now")
% "Lack of good PAKE implementations in useful languages". "Make them a hassle to use"
% few exceptions (SRP)
% new generation, getting better, standards


% One of the exception: SRP
% - "most wiidely-deployed PAKE protocol in the world"
% - TLS ciphersuite, implemented in OpenSSL
% - used extensively in iCloud Key Vault ("make SRP one of the most widely used cryptographic protocols in the world, so vast is the number of devices that Apple ships. So this is nothing to sneer at")
% - "SRP isn't the best PAKE we can deploy"

% SRP TLDR:
% - .........

% The new generation: OPAQUE



% ----- 02 Cloudflare blog post --------
% Why passwords sucks : OK

% - "Passwords are a problem"
% - "hard to remember and manage"
% - solved with password manager, but the "greater underlying problem" is not here
% - The problem is: "A password that leaves your possession is guaranteed to sacriface security, no matter its complexity or how hard it may be to guess."
% - "any direct use of a password, today, means that the password must be handled in the clear"

% - "but my password is transmitted securely over HTTPS"
% - "but I know the server stores my password in hashed form" (lot of faith in the server
% - Anyway, the password is processed in cleartext
% - accidental logging or caching !!!
% - Hardware vulnerability. "password reside in memory". "Transmitted over a shared bus to the CPU". "Attack vectors far less likely but no less severe" (Spectre, Meltdown).
% - "Ideally, servers should never see a plaintext password at all"



% Remove password : OK
% - "get rid of passwords altogether"
% - "Password-based authen is annoying"
% - "passwords are hard to remember, tedious to type, and notoriously insecure"

% - "New stadards are emerging and growing mature"
% - "Initiatives to reduce password are promising"
% - "For example, WebAuthn: web auth based on public key cryptography using hardware (or SW) tokens."

% - "But passwords are so ubiquitous that it will take a long time to agree on a supplant passwords with new standards and technology"
% - "whether their persistence is due to ease of implementation, familiarity to users, or simple ubiquity on the web and elsewhere."

% - "We'd like to make password-based auth as secure as possible while they persist."
% - For the moment, "enable a password to be useful without it ever leaving your possession"




% symmetric PAKE :

% proposed by Bellovin and Merrit in 1992: Bellovin, S. M., and Merritt, M. “Encrypted key exchange: Password-based protocols secure against dictionary attacks.” In Proc. IEEE Computer Society Symposium on Research in Security and Privacy (Oakland, May 1992), pp. 72–84.

% - " initial motivation of allowing password-authentication without the possibility of dictionary attacks based on data transmitted over an insecure channel"
% - "symmetric PAKE is a cryptographic protocol that allows two parties who share only a password to establish a strong shared secret key"
% "goals of PAKE are :"
%   1. "The secret keys will match if the passwords match, and appear random otherwise."
%   2. "Participants do not need to trust third parties (in particular, no Public Key Infrastructure),"
%   3. "The resulting secret key is not learned by anyone not participating in the protocol - including those who know the password."
%   4. "The protocol does not reveal either parties’ password to each other (unless the passwords match), or to eavesdroppers."

% - "In sum, the only way to successfully attack the protocol is to guess the password correctly while participating in the protocol"

% - In this setup, PAKE is even worse than using a classical auth method because the user's password must be stored as cleartext in the server



% asymmetric PAKE (aPAKE) :

% "only the client knows the password, and the server knows a hashed password"
% "has the 4 properties of PAKE plus one more:"
%   5. " An attacker who steals password data stored on the server must perform a dictionary attack to retrieve the password."
% - "The issue with most existing aPAKE protocols, however, is that they do not allow for a salted hash (or if they do, they require that salt to be transmitted to the user, which means the attacker has access to the salt beforehand and can begin computing a rainbow table for the user before stealing any data (SRP))."


% OPAQUE :
% - "OPAQUE is the first aPAKE protocol with a formal security proof that has this property: it allows for a completely secret salt."







% Order:

% Authentication
% 1. how to authenticate a user
% 2. attack and improvment (salt, KDF, pepper)
% 3. why password sucks (HW attack, logging, caching)
% 4. remove password (long term solution but need an alternative to secure password now)

% PAKE
% 5. PAKE at the rescue
% 6. Why PAKE have almost no adoption

% History of PAKE
% 7. Symmetric PAKE
% 8. Asymmetric PAKE and SRP
% 9. OPAQUE
% 10. KHAPE ?

\section{Problematic}
\subsection{Authentication}

\paragraph{How to authenticate a user ?}

When a user want to connect itself to a online service, he send its username (or email) for identification. Then, he need a way to prove to the server that he is indeed the person he pretend to be. This is what we call authentication. Without it, anybody can impersonate the account of someone else.

Authentication can be based on multiple factors. Something that the user knows (e.g. password, PIN, ...), something that the user has (e.g. digital certificate, OTP token device, smartphone, ...) or something that the user is (e.g. fingerprint, iris, ...). Multiple factors can be combined to obtain a strong authentication.

Traditionally, the user send the authentication value to the server through a secure channel to avoid eavesdropping and then the server compare the value that he received to the value that he store for the specific user.

This means that the server has to knows and store this sensible value before authentication (generally during register).

Traditionally on websites and softwares, passwords are used as authentication value. They are the easier to implement and the most familiar to the users.



\paragraph{Attacks and mitigations}
This setup is not ideal and can lead to multiple attacks.

In case where the server get compromised, since the server store the passwords, the attacker immediately obtain access to all passwords. This means that he can impersonate every user.

To avoid this scenario, numerous technique has been developed.
- memory-hard password hashing function (scrypt, Argon2, etc...)
- salt
- pepper

These techniques improve the security of storing password but they doesn't address a deeper problem;
When the user wants to login, he has to send its cleartext password to the server in order for the server to authenticate the user. This necessity void any password storing improvement if the server is ever persistently compromised or if password are accidentally logged or cached.


\paragraph{Why passwords are bad ?}
Passwords are a problem. They are hard to remember and to manage for the user. They are generally low-entropy and users are reusing the same passwords too often. A password manager can help to manages password but there is a greater underlying problem.
The problem is that ``a password that leaves your possession is guaranteed to sacrifice security, no matter its complexity or how hard it may be to guess. Passwords are insecure by their very existence`` \cite{PAKE_Cloudflare_blog}. % cite
Now-a-day, majority of password use require that the password is sent in cleartext.

Even if the channel between the client and the server is appropriatly secured, generally with TLS (Can also fail: PKI attack, cert missconfiguration, ... TODO), and even if on the server-side every secure password storing techniques are implemented, the password still has to be processed in cleartext.
As stated before, there can be some software issue like accidental logging or caching of the password. But hardware vulnerabilities are not to forget. While the password in processed in clear, it reside on the memory. It use a shared bus between the CPU and the memory. Hardware attacks are less likely to occur but are no less severe (Spectre, Meltdown).

In a ideal world, the server should never see the user's password in cleartext at all.


\paragraph{Get rid of password}

In summary, password are not ideal. They are difficult to remember, annoying to type and insecure.
So why don't we try to get rid of them altogether ?

Promising initiatives to reduce or remove passwords are emerging and improving. (TODO examples: WebAuthn).

These solutions are a good replacement to passwords but they require a deep change. It will take time for them to grow mature and impose themself as industry standard.

This is also because password are so ubiquitous due in part the ease of implementation and the familiarity for the users.

If we cannot get rid of passwords for now, we need a way to make it ''as secure as possible while they persist''.



This is where PAKE become interesting. It allow password-based authentication without the password leaving the client.




\subsection{Password-Authenticated Key Exchange}
\paragraph{PAKEs at the rescue}

% Password-Authentication Key Exchange (PAKE) are cryptographic primitive that allow a server to authenticate a client without ever knowing or storing client's password.
% 
% Password-authenticated key exchange (PAKE) are very powerful cryptographic primitive. They allow a server to share a key with a client or to authenticate a client without having to know or to store his password.
% For this reason, they provide better security guarantees for initializing a secure connection using password than usual mechanism where the password is transmitted to the server and then compared to a hash.


% PAKE at the rescue :

% (PAKE "provide protection for the client's password")
% "A stronger version of this guarantee can be stated as follows: after a login attempt (valid, or invalid) both the client and server should learn only whether the client’s password matched the server’s expected value, and no additional information. This is a powerful guarantee. In fact, it’s not dissimilar to what we ask for from a zero knowledge proof"
% PAKE allow to .........

% "many people dont want to run a key exchange protocol. They just want to verify that a user knows a password."
% "The great thing about PAKE is that the simpler “login only” use-case is easy to achieve. If I have a standard PAKE protocol that allows a client and server to agree on a shared key K if (and only if) the client knows the right password, then all we need add is a simple check that both parties have arrived at the same key."

% symmetric PAKE
% asymmetric PAKE

% "password auth and mutually auth key exchange in a client-server setting without relying on PKI (expect during registration)"
% "Without disclosing passwords to servers or other entities other than the client machine."
% "A secure aPAKE should provide the best possible security for a password protocol"
% only vulnerable to inevitable attacks (online guess or offline dictionary attacks if server's data get leaked)
% PKI-free

Password-Authenticated Key Exchange (PAKE) are cryptographic primitive. There is two types of PAKEs: 

\begin{itemize}
 \item Symmetric (also known as balanced) PAKE where the two party knows the password in clear
 \item Asymmetric (also known as augmented) PAKE designed for client-server scenarios. Only the client knows the password in clear
\end{itemize}

For the moment, we will focus on asymmetric PAKE (aPAKE) because it is the one that can solve our authentication problem.

aPAKE guarantee that the client's password is protected because it never leave the client's machine in cleartext.




\paragraph{Why PAKEs have almost no adoption ?}

% Why PAKE have almost no adoption :

% PAKE provides "an obvious security benefit compared to the naive approach we use to log into servers today." 
% PAKE are old. "they have been known since 1992"

% "Almost no adoption"
% "it's much easier to put a password form onto a web page than it is to do dancy crypto in the browser"
% But "even native app rarely implement PAKE for their logins"

% patents ("though most of these are expired now")
% "Lack of good PAKE implementations in useful languages". "Make them a hassle to use"
% few exceptions (SRP)
% new generation, getting better, standards


\end{document}

